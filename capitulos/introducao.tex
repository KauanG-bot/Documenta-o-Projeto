\chapter{Introdução}
\label{ch:introducao}
\begin{resumocapitulo}
	Esses textos são recursos valiosos para compreender tanto o frontend quanto o backend de um projeto, facilitando que qualquer pessoa que consulte a documentação possa realizar melhorias e alterações nas páginas. Eles proporcionam uma compreensão clara do funcionamento do site, permitindo que os usuários sigam adiante com atualizações, correções de erros e aprimoramentos. Esse tipo de texto breve também capacita os usuários a navegarem no site com conhecimento prévio, contribuindo para uma construção mais eficiente do projeto. Além disso, é uma oportunidade para aplicar linguagens já conhecidas da faculdade e aprender novas tecnologias que possam aprimorar ainda mais a programação.
\end{resumocapitulo}


\label{sec:citacoes}

\section{Front End}
\subsection{Citação Direta Front End}
\label{subsec:citacao_direta}
 Home - Inserido nome e logo da empresa, cabecalho, Artigos (Quando clicado foi ser direcionado para baixo com um scrolldown para as paginas de artigo), Post (será encaminhado para pagina de postagens, About Us (envaminhado para pagina sobre nós  que conta sobre quem somos, com o que trabalham,
Falando sobre as funcionalidades da pagina home, temos interface front para o Sing-Up e Login de usuário, mostrando um pouco sobre a empresa e fazendo a pessoa querer conhecer a pagina
\vspace{12pt}

Article Python Pandas - Falando um pouco sobre python e como é utilizada a biblioteca pandas dentro do mesmo, artigo para informar o usuário e auxiliar na produção em python.

Article Kotlin - informando o usuário sobre Kotlin onde é utilizado para que e como os códigos nesse formato auxiliam na programação mobile

Article SQL - Informando o usuário sobre SQL (Structured Query Language) explicando também como funciona um banco relacional o para que serve e em que impacta em dados para uma empresa

About Us - Falando um pouco sobre os criadores do site e da onde vem essa ideia e para que isso ajuda  o usuário que gosta de programaçao

Post - Pagina de postagem do usuário, parte construída para manter uma comunidade de programação ativa, onde programadores e não programadores se ajudem se auxiliem com posts voltados para programação gente



\section{Back End}
\subsection{Citação Direta Back End}
\label{subsec:citacao_indireta}
No desenvolvimento do Back-End fizemos algumas pesquisas para escolhermos as melhores linguagens para  desenvolver nosso projeto; após algumas reuniões decidimos utilizar as linguagens: PHP, JavaScript e SQL. 
 Na Página “Home” houve uma integração com o banco de dados para efetuarmos o cadastro dos usuários, assim como, possibilitar o login.
Além disso, utilizamos a linguagem JavaScript para facilitar o acesso e o encerramento dos elementos de pop-up destinados ao cadastro de usuários e ao processo de login, desse modo, isso faz com que o usuário se comunique indiretamente como o PHP e o SQL(banco de dados). Ademais, com a criação de uma aba direcionada para que realizem postagens possui uma funcionalidade que permite que usuário faça um upload de uma imagem direto do seu desktop, salvando no banco de dados e, consequentemente, o conteúdo fica disponível para que todos os usuários consigam visualizar. Sob essa ótica, firmamos uma conexão com o Banco de Dados para obtermos acesso às informações armazenadas e podermos armazenar novas. Com o acesso realizado salvamos cada post de acordo com o contrato estabelecido pelo próprio banco e fornecemos uma liberação indireta para que os usuários insiram seus posts e visualizem os dados salvos.