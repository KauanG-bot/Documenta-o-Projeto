\chapter{Fundamentação Teórica}
\label{ch:identificador}
	\begin{resumocapitulo}
		Apresentar quais foram as pesquisas, fontes, ideias que tivemos para o desenvolvimento do projeto
	\end{resumocapitulo}

	\section{Visão Geral}

            Nós do time utilizamos vídeos no YouTube, assim como revisamos estudos e projetos do semestre anterior para nos ajudar. Alguns de nós já tinham alguma experiência com bancos de dados, o que acabou facilitando o processo de criação para nós. Nossa meta era entender melhor os códigos e aprofundar nosso conhecimento através da documentação das linguagens. Além disso, nos aventuramos a explorar conexões com PHP e implementamos pop-ups de login e cadastro. Aprendemos bastante com vídeos e também com as experiências que tivemos no semestre anterior.
            \vspace{12pt}
            Durante o processo, realizamos diversos testes, até mesmo migrando entre diferentes bancos de dados como MySQL e SQL Server. Lidamos com questões relacionadas à tipagem de dados e outras questões técnicas. Também experimentamos a viabilidade de usar o Figma para realizar algumas funções de design para reproduzir no HTML e CSS. Nosso objetivo era sempre buscar a melhor abordagem para o nosso código e para o projeto como um todo.

